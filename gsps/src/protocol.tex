\documentclass[11pt]{article}
\usepackage{amsmath}
\usepackage{amssymb}
\usepackage{geometry}
\usepackage{graphicx}
\geometry{margin=1in}

\title{GSPS: Geometric Semantic Positioning System}
\author{JB Larson}
\date{\today}

\begin{document}

\maketitle

\begin{abstract}
I present the Geometric Semantic Positioning System (GSPS), a framework that reveals semantic structure by projecting high-dimensional embeddings onto constrained 2D geometric templates. Using 1024-dimension Alibaba-NLP/gte-large-en-v1.5 embeddings of Wikipedia articles, the GSPS system uses relative positioning to place inner nodes according to relative semantic-similarity to fixed anchor nodes.

I introduce four primitive types: point (1-anchor), edge (2-anchor), triangle (3-anchor), and square (4-anchor).

The protocol enables deterministic, framework-agnostic semantic positioning where any system with the same anchor configuration will compute identical coordinates for a given concept.
\end{abstract}

\section{Core Principles}

\subsection{Anchor-Defined Spaces}

Semantic space is bounded by 1--4 fixed anchor nodes positioned at predetermined coordinates:

\begin{itemize}
    \item \textbf{4-Anchor:} Square configuration $[(-1,0), (1,0), (0,1), (0,-1)]$
    \item \textbf{3-Anchor:} Triangle configuration $[(-1,0), (1,0), (0,1)]$
    \item \textbf{2-Anchor:} Edge configuration $[(-1,0), (1,0)]$
    \item \textbf{1-Anchor:} Point configuration $[(0,0)]$
\end{itemize}

\subsection{Force-Based Positioning}

Inner nodes (non-anchors) are positioned through:

\begin{itemize}
    \item Gravitational attraction when cosine similarity to an anchor exceeds the dynamic pivot
    \item Electromagnetic repulsion when cosine similarity falls below the dynamic pivot
\end{itemize}

The dynamic pivot is calculated as the mean cosine similarity across all anchors, enabling manifold-dependent relative positioning.

\subsection{Protocol Invariance}

Two systems implementing GSPS with identical:

\begin{itemize}
    \item Anchor node vectors
    \item Anchor coordinates
    \item Force parameters ($\alpha$, $\gamma$, $\beta$)
\end{itemize}

will produce identical coordinates for any given inner node vector.

\section{Mathematical Framework}

\subsection{Dynamic Pivot Calculation}

The dynamic pivot $\mu$ is the mean cosine similarity between the center node and all anchors:

$$\mu = \frac{1}{n} \sum_{i=1}^{n} \cos(v_c, v_{a_i})$$

where $v_c$ is the center vector, $v_{a_i}$ is anchor $i$'s vector, and $\cos(a,b) = \frac{a \cdot b}{\|a\| \|b\|}$.

\subsection{Force Computation}

For each anchor $i$, compute the deviation from pivot:

$$\delta_i = \cos(v_c, v_{a_i}) - \mu$$

The polarity determines attraction ($\delta_i \geq 0$) or repulsion ($\delta_i < 0$):

$$\rho_i = \text{sgn}(\delta_i)$$

Force magnitude scales with deviation:

$$m_i = \beta + \gamma |\delta_i|^\alpha$$

Force vector:

$$\mathbf{F}_i = \mathbf{p}_i \cdot m_i \cdot \rho_i$$

where $\mathbf{p}_i = (x_i, y_i)$ is the anchor's fixed coordinate.

\subsection{Position Resolution}

$$\mathbf{r} = \frac{1}{n} \sum_{i=1}^{n} \mathbf{F}_i$$

Final coordinates are normalized:

$$\mathbf{r}_{\text{final}} = \tanh(\kappa \cdot \mathbf{r})$$

with $\kappa = 1.2$ (stiffness), ensuring $r_j \in [-1, 1]$ for $j \in \{x, y\}$.


\begin{figure}[p]
\section{Visual Examples}
\centering
\includegraphics[width=0.8\linewidth]{gsps_v2_square.png}
\\
\small 4-Anchor Square

\centering
\includegraphics[width=0.8\linewidth]{gsps_v2_triangle.png}
\vspace{0.5em}
\\
\small 3-Anchor Triangle
\end{figure}

\begin{figure}[p]
\centering
\includegraphics[width=0.9\linewidth]{gsps_v2_edge.png}
\\
\small 2-Anchor Edge

\centering
\includegraphics[width=0.9\linewidth]{gsps_v2_point.png}
\vspace{0.5em}
\\
\small 1-Anchor Point
\end{figure}


\end{document}
